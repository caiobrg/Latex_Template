% In: settings/packages.tex

%======================================================================
% 1. LANGUAGE AND FONT ENCODING
% Essential for correct hyphenation, character display, and quotes.
%======================================================================
\usepackage[portuguese]{babel} % Language-specific settings for Portuguese
\usepackage[utf8]{inputenc}     % (Still good practice for compatibility)
\usepackage[T1]{fontenc}        % Modern font encoding for output
\usepackage{csquotes}           % Context-sensitive quotation marks

%======================================================================
% 2. TYPOGRAPHY AND LAYOUT
% Controls the look and feel of the page and text.
%======================================================================
\usepackage{libertine}     % For the Libertine font family
\usepackage[final, nopatch=footnote]{microtype} % Improves typography (justification, spacing)

% --- PAGE GEOMETRY (REPLACES vmargin) ---
% Modern and powerful way to control margins and page layout.
\usepackage[
  left=3cm,
  right=3cm,
  top=2.5cm,
  bottom=2.5cm,
  headheight=1.5cm % Adjust as needed for your header
]{geometry}

% --- PARAGRAPH STYLE ---
% CHOOSE ONE of the following two styles.
% Style A: Space between paragraphs, no indent (parskip is active).
\usepackage[skip=5pt]{parskip}
% \usepackage{indentfirst} % DO NOT use with parskip.

% Style B: Indented paragraphs (the LaTeX default).
% To use this, COMMENT OUT the \usepackage{parskip} line above,
% and UNCOMMENT the \usepackage{indentfirst} line below.
% \usepackage{indentfirst}

% --- HEADERS AND FOOTERS ---
\usepackage{fancyhdr}

%======================================================================
% 3. MATH AND SYMBOLS
%======================================================================
\usepackage{amsmath, amssymb} % Essential AMS math packages
\usepackage{breqn}             % For automatic line breaking of equations
\usepackage{steinmetz}         % For phasors and other engineering notation

%======================================================================
% 4. TABLES AND FIGURES
%======================================================================
\usepackage{graphicx}     % For including images (\includegraphics)
\usepackage{booktabs}     % For professional-looking tables (\toprule, \midrule)
\usepackage{tabularx}     % For tables with auto-adjusting column widths
\usepackage{multicol}     % For multi-column text
\usepackage{subcaption}   % For subfigures (e.g., Figure 1a, 1b)
\usepackage{float}        % Improved control over float placement [H]
\usepackage{makecell}     % For line breaks and advanced formatting within table cells

%======================================================================
% 5. DOCUMENT STRUCTURE AND REFERENCES
%======================================================================
% For bibliography in the table of contents
\usepackage[numbib, nottoc, notlof]{tocbibind}
\usepackage{pdfpages} % To include external PDF files

%======================================================================
% 6. CODE LISTINGS
%======================================================================
\usepackage{listings}
\usepackage{xcolor}   % Required for colored listings (and other color needs)

%======================================================================
% 7. CITATIONS
%======================================================================
\usepackage[
  backend=biber,
  style=nature,
]{biblatex}

%======================================================================
% 8. HYPERLINKS AND URLS (LOAD LAST)
%======================================================================
\usepackage{xurl} % Allows line breaks in URLs anywhere. Better than the default.
\usepackage{hyperref}
\usepackage{bookmark} % Improves PDF bookmark generation with hyperref

